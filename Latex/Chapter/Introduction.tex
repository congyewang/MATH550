\section{Introduction}
According to Feu (n.d.) published in Mathematics Education Innovation, blackbirds have unstable body weight, sometimes less than 90g, sometimes more than 130g. They can change their weight according to conditions and gain weight on cold nights to burn approximately 5\% of their body mass for warmth. At the end of cold conditions, they lose weight and fly faster to avoid predators. The way to assess the standard size of a blackbird is to measure the length of its wings which is the distance from the carpal joint to the wingtip.

The dataset used in this analysis is 25 years of blackbirds captured in the same garden in East Midland, including foreign birds with foot rings and a small number of native British birds. When the birds are first caught, they are put on foot rings so that they can clearly trace back to the date of the first capture.

Some birds have been repeatedly trapped, according to the data requirements, this part of the data needs to be ignored. There are some missing values in the dataset, including age, gender, wing length and weight.
