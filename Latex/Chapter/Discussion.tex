\section{Discussion}

In this paper, the relationship between wing length and sex and age of blackbirds was quantitatively analyzed by the OLS model. The results showed that there was a positive correlation between age and wing length and males tend to have longer wings than females. The disadvantage of this paper is that it does not consider the change of time, which means that the influence of time factor on the length of blackbird wing is reflected in $\epsilon$. Since the length of blackbirds' wing has periodic time accumulation effect, the residual term must be related to its lag term, which violates the assumption of OLS estimation that the residual term is independent and identically distributed, and the result of model estimation does not conform to the excellent property of blue in Gaus-Markov theorem. In addition, the processing of the missing value is also relatively simple, there is no hierarchical interpolation fitting.
