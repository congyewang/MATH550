\section*{Appendix}
\subsection{Responding to Peer Review}
The use of keywords is a useful way to aid the readers understanding. A glossary in the appendix to explain the key words would go even further to help allow non- mathematically minded people understand the paper. When using acronyms, it is useful to filly define them once before using them for the rest of the paper (What is OLS?).

Mentioning the omission of repeated measurements of a blackbird is useful in making the results reproducable. It is worth discussing how you did this and how it may effect the results. Cleaning the data according to species is a good decision, but a reference to how ring numbers are linked to species would be useful for the reader.

The model and conclusions are explained well and the IMRAD structure is well followed. To aid reader understanding in variable correlation, plots could be used. The model is well shown, but the use of the error term should be explained.

\begin{itemize}
	\item \textcolor[rgb]{1,0,0}{Thank you for your suggestions on the shortcomings of my article. For the OLS abbreviation, I have revised the introduction section. For the explanation of the error item, I put it in the discussion section. Limited by the length of the article, I put the visualization results and data cleaning code into the appendix.}
\end{itemize}

\newpage

Thank you for submitting your report; I really enjoyed reading it! Also, I like how you analysed the data. Below, I wrote some comments that I hope will help you improve your report.

I think that in the abstract there is slightly too much emphasis on the method of analysis. You may want to consider shortening the method section (of the abstract) and expanding a little on the background and what you found, and what the results mean. You can put the details concerning the dataset you used in the method section. To make space for this, I would recommend removing the last subsection of Table 1 (the omnibus, skew, and kurtosis part).

In the introduction, you could mention briefly some literature concerning blackbirds’ size in relation to their sex and age. It may also be useful, if you have space, to mention what you expect the results to be given the literature (or state hypotheses). Here are links to some references that may be useful: \url{https://onlinelibrary.wiley.com/doi/abs/10.1111/j.0030-1299.2006.14183.x https://doi.org/10.1080/00063656309476036 http://www.jstor.org/stable/2407788}

You mentioned that the independent variables do not explain variation in the dependent variable well, but, in the context of the blackbirds dataset (where the measurements were taken in a natural setting with no control over extraneous variables), I would consider the variation in the independent variables to explain the variation in the dependent variable relatively well. Specifically, in this case, I think that if the predictor variables ac- count for 45\% of the variance in the outcome variable, the predictors explain a substantial and meaningful proportion of the variance in the outcome. Overall, well-done.

\begin{itemize}
	\item \textcolor[rgb]{1,0,0}{Thank you for your advice and encouragement. For the results in the table, I will delete it. As well as, thank you for your supplementary references. I checked it and it really helped a lot. About the final question, I cannot point out this in the report due to the space limitation. I suppose that if the missing important explanatory variable has nothing to do with other explanatory variables, that is to say, the explanatory variable is orthogonal to the other explanatory variables, then the variance of unbiased estimator of parameter estimator is also unbiased, but it will cause biased estimation of constant term.}
\end{itemize}
