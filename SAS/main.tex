\documentclass[a4paper, 12pt]{article}

%---------------------------------------------------------%
%------------------ Document Preamble --------------------%
%---------------------------------------------------------%
%-------------------------------------------%
%---- First load the necessary packages ----%
%-------------------------------------------%
\usepackage{longtable}
\usepackage{booktabs}
\usepackage{graphicx}
% For including graphics (via ncludegraphics).
\usepackage{amsmath}
% Improves typographic quality of mathematical output.
\usepackage{amsfonts}
% For mathematical fonts.

% Package for no indentation and to include a line between paragraphs.
\usepackage[parfill]{parskip}

% Set the layout of the document.
\usepackage[left=2.5cm,right=2.5cm, top=1.5cm,bottom=1.5cm]{geometry}

\usepackage{amsthm} % Needed to typeset theorem environments.
\usepackage[affil-it]{authblk} % Needed for author and affiliation.
\usepackage{amsopn} % Allows the declaration of new mathematical operators.
\usepackage{amssymb} % Extended set of mathematical symbols.
\usepackage{mathrsfs} % Raph Smith's mathematical script font.
\usepackage{booktabs} % Improves the typographical quality of tables.
\usepackage{natbib} % Citations and bibliography.
\usepackage{tikz} % Extends the figure generation capabilities of LaTeX.
\usepackage{algorithm} % Necessary for including algorithmic structures.
\usepackage{float} % Allows the creation of own floating environments.
\usepackage{caption} % Allows the customisation of captions.
\usepackage{abstract} % For including abstracts in documents.
\usepackage{listings} % For the inclusion of source code.
\usepackage{color} % Used for colour commands.
\usepackage{pgfplots} %extends tikz functionality

% Automatic inclusion of hyperlinks with back references.
\usepackage[pagebackref]{hyperref}

%-------------------------------------------%

% Making a float for an algorithm.
\newfloat{algorithm}{t}{lop}
\renewcommand{\thealgorithm}{\arabic{section}.\arabic{algorithm}}

% Numbers equations by section.
\numberwithin{equation}{section}

% Add biliography
% \usepackage[style=apa,backend=biber]{biblatex}
% \usepackage[backend=biber]{biblatex}
% \addbibresource{ref.bib}
% Set biliography style.
% \bibliographystyle{apalike}

% Commands for special set of numbers.
\newcommand{\NN}{\mathbb{N}}
\newcommand{\ZZ}{\mathbb{Z}}
\newcommand{\RR}{\mathbb{R}}
\newcommand{\PP}{\mathbb{P}}
\newcommand{\EE}{\mathbb{E}}

% Commands for useful operators.
\newcommand{\di}{\text{diag}}
\newcommand{\de}{\mathrm{d}}
\newcommand{\si}{\Sigma}
\newcommand{\del}{\Delta}
\newcommand{\cov}[1]{\, {\rm cov}\left( #1 \right) }
\newcommand{\var}[1]{\, {\rm var}\left( #1 \right) }
\newcommand{\toas}{\xrightarrow{\text{a.s.}}}
\newcommand{\todis}{\xrightarrow{\text{d}}}
\newcommand{\supp}{\text{supp}}
\newcommand{\doubint}{\!\int\!\!\!\int}

% Command for text within a math sub/super-script.
\newcommand{\stext}[1]{\text{\scriptsize{#1}}}

% Command for highlighting pieces of text
\newcommand{\onote}[1]{\colorbox{orange}{#1}}

% Theorem definitions/
\newtheorem{lemma}{Lemma}[section]
\newtheorem{theorem}[lemma]{Theorem}
\newtheorem{proposition}[lemma]{Proposition}
\newtheorem{corollary}[lemma]{Corollary}

\theoremstyle{definition}
\newtheorem{definition}{Definition}[section]

\theoremstyle{remark}
\newtheorem{remark}{Remark}[section]
\newtheorem{example}[remark]{Example}

% An environment for centre aligning equations which extend past the text width.
\newenvironment{longeq}
 {\begin{displaymath}\begin{lrbox}{\overlongequation}$\displaystyle}
 {$\end{lrbox}\makebox[0pt]{\usebox{\overlongequation}}\end{displaymath}}

% Listings options.

\lstset{ %
basicstyle=\footnotesize,       % the size of the fonts that are used for the code
backgroundcolor=\color{white},  % choose the background color. You must add \usepackage{color}
showspaces=false,               % show spaces adding particular underscores
showstringspaces=false,         % underline spaces within strings
showtabs=false,                 % show tabs within strings adding particular underscores
frame=single,           % adds a frame around the code
tabsize=2,          % sets default tabsize to 2 spaces
captionpos=b,           % sets the caption-position to bottom
breaklines=true,        % sets automatic line breaking
breakatwhitespace=false,    % sets if automatic breaks should only happen at whitespace
escapeinside={\%*}{*)}          % if you want to add a comment within your code
}

%-------------------------------------------%
%------------- File Information ------------%
%-------------------------------------------%
\title{\textbf{35427962}}
\author{}
\date{}
\begin{document}
\maketitle
%-------------------------------------------%
%--------------- File Start ----------------%
%-------------------------------------------%
\begin{abstract}
The purpose of this report is to provide answers to questions 1 through 6 and to include descriptive tables and figures on which the answers are based on, and to discuss question 1 in detail using AR(1) model and provide the corresponding explanations. In conclusion, the answer to the first four questions is d, b, a, and c. The ID of 1 and 2 may suffer from high frequency attacks. The effect of the control and restraint training is not predicted to reduce the number of attacks, but it may do the opposite. However, control and restraint training may reduce the lethality of attacks.
\end{abstract}

\section*{Question I}
\subsection*{Model Fitting}
This model uses time series to quantify changes in violence over months over months. The dataset is selected and merged into groups at first, and the indicator of score given to incident is summed by month. Firstly, the graph of sequence diagram is ploted.

\begin{figure}[H]
	\centering
		\includegraphics[scale=0.25]{Pic/Q1/1.png}
	\caption{Sequence Diagram}
	\label{f1}
\end{figure}

Based on figure of \ref{f1}, the sequence has no significant non-stationary features. White noise detection shows that there is a correlation between the sequence values, which is a non white noise sequence.

Secondly, I use three methods to solve the best parameters, namely MINIC, SCAN, and ESACF. The results of them are shown in table \ref{t1}.

\begin{table}[H]
	\centering

\begin{longtable}{rrrrrr}
\toprule
	\multicolumn{3}{c}{SCAN} &    \multicolumn{3}{c}{ESACF}\\
	\midrule
    p+d &    q &    BIC &    p+d &    q &    BIC\\
\endhead
    1 &    0 &    5.405064 &    1 &    0 &    5.405064\\
    0 &    1 &    5.60567 &    0 &    1 &    5.60567\\
     &      &      &    4 &    0 &    5.152248\\
     &      &      &    5 &    0 &    5.066185\\
\bottomrule

\caption{ARMA(p+d,q) Tentative Order Selection Tests}
\label{t1}
\end{longtable}
\end{table}

The results show that MA(5) model should be chosen. The ARIMA module of SAS is used to implement the model. After adjusting the parameters to P = 0 and Q = 5, the model results are shown in the following figure \ref{t2}.

\begin{table}[H]
\centering
\begin{longtable}{lrrrrr}
\toprule
   Parameter &    Estimate &    Standard {\newline} Error &    t~Value &    Approx {\newline} Pr > |t| &    Lag\\
\endhead
\midrule
   MU &    35.06369 &    8.46743 &    4.14 &    0.0005 &    0\\
   MA1,1 &    $-$0.69146 &    0.21930 &    $-$3.15 &    0.0048 &    1\\
   MA1,2 &    $-$0.28225 &    0.26805 &    $-$1.05 &    0.3043 &    2\\
   MA1,3 &    0.04334 &    0.28187 &    0.15 &    0.8793 &    3\\
   MA1,4 &    $-$0.02787 &    0.27443 &    $-$0.10 &    0.9201 &    4\\
   MA1,5 &    0.01157 &    0.22909 &    0.05 &    0.9602 &    5\\
\bottomrule
\caption{Conditional Least Squares Estimation of MA(5)}
\label{t2}
\end{longtable}

\begin{longtable}{lr}
\toprule
   Constant Estimate &    35.06369\\
   Variance Estimate &    552.6743\\
   Std Error Estimate &    23.50903\\
   AIC &    252.3359\\
   SBC &    260.111\\
   Number of Residuals &    27\\
\bottomrule
\caption{Model Diagnosis of MA(5)}
\label{t3}
\end{longtable}
\end{table}

According to the table \ref{t2}, we found that only two coefficients are significant. Therefore, the model should be redraft. After I calculated 121 models with P from 0 to 10 as well as Q between 0 and 10, AR(1) need to be considered since the AIC is the lowest in these models. The result of the models was given as follow.

\begin{table}[H]
\centering
\begin{longtable}{lrrrrr}
\toprule
   Parameter &    Estimate &    Standard {\newline} Error &    t~Value &    Approx {\newline} Pr > |t| &    Lag\\
\endhead
\midrule
   MU &    32.87074 &    9.17622 &    3.58 &    0.0014 &    0\\
   AR1,1 &    0.56566 &    0.16798 &    3.37 &    0.0025 &    1\\
\bottomrule
\caption{Conditional Least Squares Estimation of AR(1)}
\label{t4}
\end{longtable}
\end{table}

\begin{table}[H]
\centering
\begin{longtable}{lr}
\toprule
   Constant Estimate &    14.27693\\
   Variance Estimate &    495.1036\\
   Std Error Estimate &    22.25092\\
   AIC &    246.0734\\
   SBC &    248.6651\\
   Number of Residuals &    27\\
\bottomrule
\caption{Model Diagnosis of AR(1)}
\label{t5}
\end{longtable}
\end{table}

Comparing table \ref{t3} and \ref{t5}, we obtain the conclusion that AR(1) is much more appropriate than MA(5). In addition,, the parameters of AR(1) are significant since p-values are both less than 0.05 (Figure \ref{f2}).


\begin{figure}[H]
\centering
\includegraphics[scale=0.25]{Pic/Q1/2.png}
\caption{Trend and Correlation Analysis for Sum of Score Given to Incident}
\label{f2}
\end{figure}

Thirdly, according to the autocorrelation coefficient in figure \ref{f2}, except for the autocorrelation coefficients of the first order delay being instead of are in the range of twice standard deviation, the autocorrelation coefficients of other orders fluctuate in the range of twice standard deviation. Based on this feature, we can judge that the sequence has short-term correlation, and further confirm that the sequence is stable. Furthermore, the process of the partial autocorrelation coefficient decaying to zero is further investigated. Except the first order partial autocorrelation coefficient being instead of is in the range of double standard deviation, the other order partial autocorrelation coefficients are in the range of double standard deviation, which is a typical characteristic of the first-order truncated partial autocorrelation coefficient.

Finally, according to the table \ref{t4}, the fomula of AR(1) can be demonstrated as follow.

\begin{equation}
x_{t} = 32.87074 + 0.56566\times x_{t-1} + \epsilon_{t}, Var(\epsilon_{t}) = 495.1036
\label{e1}
\end{equation}

where $x_{t}$ is sum of score given to incident by month t.

\begin{figure}[H]
\centering
\includegraphics[scale=0.5]{Pic/Q1/3.png}
\caption{Residual Correlation Diagnostics for Sum of Score Given to Incident}
\label{f3}
\end{figure}

\begin{figure}[H]
\centering
\includegraphics[scale=0.25]{Pic/Q1/4.png}
\caption{Residual Normality Diagnostics for Sum of Score Given to Incident}
\label{f4}
\end{figure}

These two figures (\ref{f2}, \ref{f3}) show that the effect of the model is good. The residual error of the model is in line with normal distribution from the chart of QQ plot.

\subsection*{Hypothesis testing}

The significance test of model \ref{e1} is white noise test of residual sequence. Therefore, the null hypothesis and alternative hypothesis are:

$$H_{0}: \rho_{1} = 0, \forall m \ge 1$$
$$H_{1}: \exists \rho_{k}\neq 0, \forall m \ge 1, k\le m$$

The test statistic is LB (Ljung-Box):

$$LB = n(n + 2) \sum_{k=1}^{m}(\frac{\hat{\rho_{k}}^{2}}{n-k})\sim \chi^{2}(m), \forall m > 0$$

If the null hypothesis is rejected, it means that there are still some relevant information in the residual sequence, and the fitting model is not significant. If the original hypothesis cannot be rejected, the fitting model is considered significant. The result is given as table \ref{tlb}.

\begin{table}[H]
\begin{longtable}{rrrrrrrrrr}
\toprule
   To Lag &    Chi-Square &    DF &    Pr > ChiSq &    \multicolumn{6}{c}{Autocorrelations}\\
  \midrule
\endhead
   6 &    10.73 &    6 &    0.0969 &    0.551 &    0.155 &    $-$0.032 &    0.002 &    0.036 &    0.142\\
\bottomrule
\caption{Autocorrelation Check for White Noise}
\label{tlb}
\end{longtable}
\end{table}

It can be seen from the results that p-value is 0.0969, which is greater than 0.05. Therefore, we cannot reject the null hypothesis. This is a good proof that the data is a very stable white noise sequence. In conclusion, this model proves the fact that the trends in violence is non-monotonic over time. The option D may be the correct answer.


\section*{Question II}

\begin{figure}[H]
\centering
\includegraphics[scale=0.25]{Pic/Q2/1.png}
\caption{The Rate of Attacks in Each Category}
\label{f5}
\end{figure}

At the outset, we can see that it is not the same throughout the 30-month period by the figure \ref{f5} above.

\begin{figure}[htbp]
\centering
\begin{minipage}[t]{0.48\textwidth}
\centering
\includegraphics[width=6cm]{Pic/Q2/2.png}
\caption{The Percentage of Near Miss}
\label{f6}
\end{minipage}
\begin{minipage}[t]{0.48\textwidth}
\centering
\includegraphics[width=6cm]{Pic/Q2/3.png}
\caption{The Percentage of Assault}
\label{f7}
\end{minipage}
\end{figure}

Moreover, these two figures (\ref{f6}, \ref{f7}) show that is not always
proportional to the number of attacks each month.

\begin{figure}[H]
\centering
\includegraphics[scale=0.5]{Pic/Q2/4.png}
\caption{The Rate of Attacks in Each Category for the Second Half
of the 30 Months}
\label{f8}
\end{figure}
Finally, figure (\ref{f8}) shows that there was a higher proportion of less threatening incidents in the second half
of the 30 months. In conclusion, these figures demonstrate the fact that less threatening incidents accounts a higher proportion. The option B may be the correct answer.


\section*{Question III}

I used ANOVA model to compare the influence of gender on violence at different levels. Table \ref{t9} shows the model output.

\begin{table}[H]
\centering
\begin{longtable}{lrrrrr}
\toprule
   Source &    DF &    Sum of Squares &    Mean Square &    F Value &    Pr~>~F\\
\endhead
\midrule
   Model &    6 &    2806.376471 &    467.729412 &    Infty &    <.0001\\
   Error &    163 &    0.000000 &    0.000000 &      &     \\
   Corrected Total &    169 &    2806.376471 &      &      &     \\
\bottomrule
\end{longtable}

\begin{longtable}{rrrr}
\toprule
   R-Square &    Coeff Var &    Root MSE &    Score given to incident~Mean\\
\endhead
\midrule
   1.000000 &    0 &    0 &    5.388235\\
\bottomrule
\end{longtable}

\begin{longtable}{lrrrrr}
\toprule
   Source &    DF &    Anova SS &    Mean Square &    F Value &    Pr~>~F\\
\endhead
\midrule
   Sex*Category &    6 &    2806.376471 &    467.729412 &    Infty &    <.0001\\
\bottomrule
\end{longtable}

\caption{ANOVA of the Interaction between Sex of Perpetrator and Category of Incident}
\label{t6}
\end{table}

\begin{table}[H]
\centering
\begin{longtable}{llrrr}
\toprule
   Level of {\newline} Sex of Perpetrator &    Level of {\newline} Category of Incident &    N &    \multicolumn{2}{c}{Score Given to Incident}\\

   ~ &    ~ &    ~ &    Mean &    Std Dev\\
\endhead
\midrule
   F &    1 &    37 &    2.0000000 &    0\\
   F &    2 &    31 &    5.0000000 &    0\\
   F &    3 &    30 &    10.0000000 &    0\\
   M &    1 &    36 &    2.0000000 &    0\\
   M &    2 &    19 &    5.0000000 &    0\\
   M &    3 &    12 &    10.0000000 &    0\\
   M &    4 &    5 &    20.0000000 &    0\\
\bottomrule
\caption{Mean of the Interaction between Sex of Perpetrator and Category of Incident}
\label{t7}

\end{longtable}
\end{table}

\begin{figure}[H]
\centering
\includegraphics[scale=0.25]{Pic/Q3/3.png}
\caption{ANOVA of the Interaction between Sex of Perpetrator and Category of Incident}
\label{f9}
\end{figure}

According to the table \ref{t6}, we can see p-value is less than 0.0001. Therefore, it is sure that gender has an impact on different levels of violence. The results showed that men were more likely to cause fatal attacks. The option A may be the correct answer.




\section*{Question IV}

\begin{figure}[H]
\centering
\includegraphics[scale=0.25]{Pic/Q4/1.png}
\caption{Relative Frequencies of Victim Grade}
\label{f10}
\end{figure}

As can be seen from figure \ref{f10}, the relative frequency of the victim's grade changes over time. As a result, the option C may be chosen.

\section*{Question V}

\begin{figure}[H]
\centering
\includegraphics[scale=0.5]{Pic/Q5/1.png}
\caption{Freauency of Particular Individuals Attacked}
\label{f11}
\end{figure}

The ID of 1 and 2 received the high frequency of attacks from figure \ref{f11}. Therefore, I would like to select option A.


\section*{Question VI}

\begin{table}[H]
\centering
\begin{longtable}{lrrrrrrr}
\toprule
   Parameter &    DF &    Estimate &    Std. &    \multicolumn{2}{c}{Wald 95\% Confidence Limits} &    Wald $\chi^{2}$ &    Pr~>~$\chi^{2}$\\
\endhead
\midrule
   Intercept &    1 &    1.1835 &    0.2542 &    0.6853 &    1.6816 &    21.68 &    <.0001\\
   CRStaff &    1 &    0.0257 &    0.0169 &    $-$0.0074 &    0.0588 &    2.32 &    0.1280\\
   Scale &    0 &    1.0000 &    0.0000 &    1.0000 &    1.0000 &      &     \\
\bottomrule
\caption{Analysis of Maximum Likelihood Parameter Estimates}
\label{t8}
\end{longtable}

\begin{longtable}{lrrr}
\toprule
   Criterion &    DF &    Value &    Value/DF\\
\endhead
\midrule
   Deviance &    33 &    236.4986 &    7.1666\\
   Scaled Deviance &    33 &    236.4986 &    7.1666\\
   Pearson Chi-Square &    33 &    363.4218 &    11.0128\\
   Scaled Pearson X2 &    33 &    363.4218 &    11.0128\\
   Log Likelihood &      &    90.4709 &     \\
   Full Log Likelihood &      &    $-$167.2711 &     \\
   AIC (smaller is better) &      &    338.5421 &     \\
   AICC (smaller is better) &      &    338.9171 &     \\
   BIC (smaller is better) &      &    341.6528 &     \\
\bottomrule
\caption{Criteria for Assessing Goodness of Fit}
\label{t9}
\end{longtable}
\end{table}

I use Poisson's generalized linear model to fit the number of attacks and the number of staffs who are received the control and restraint training. The results are shown in the table (\ref{t8}, \ref{t9}). It can be clearly seen that the more people are trained, the more times they attack.

\begin{table}[H]
\centering
\begin{longtable}{lrrr}
\toprule
   Test &    Chi-Square &    DF &    Pr~>~ChiSq\\
\endhead
\midrule
   Likelihood Ratio &    25.3116 &    1 &    <.0001\\
   Score &    23.9340 &    1 &    <.0001\\
   Wald &    23.0210 &    1 &    <.0001\\
\bottomrule
\caption{Testing Global Null Hypothesis: BETA=0}
\label{t10}
\end{longtable}

\begin{longtable}{llrrrrr}
\toprule
   Parameter &    ~ &    DF &    Estimate &    Standard {\newline} Error &    Wald {\newline} Chi-Square &    Pr~>~ChiSq\\
\endhead
\midrule
   Intercept &    2 &    1 &    $-$2.2009 &    0.4307 &    26.1110 &    <.0001\\
   Intercept &    5 &    1 &    $-$0.7888 &    0.3972 &    3.9441 &    0.0470\\
   Intercept &    10 &    1 &    1.8759 &    0.5565 &    11.3625 &    0.0007\\
   CRStaff &      &    1 &    0.1560 &    0.0325 &    23.0210 &    <.0001\\
\bottomrule
\caption{Analysis of Maximum Likelihood Estimates}
\label{t11}
\end{longtable}
\end{table}

\begin{figure}[H]
\centering
\includegraphics[scale=0.5]{Pic/Q6/1.png}
\caption{Predicted Cumulative Probabilities for Score Given to Incident}
\label{f12}
\end{figure}

The logistic model is used to fit the relationship between the severity of attack and the number of trained employees. According to the figure \ref{f12}, the more people are trained, the more obvious the probability of low-level injuries. This means that increasing the number of trainees will reduce the occurrence of high-level attacks, but not the number of attacks.


\clearpage

\section*{Appendix}
\subsection{Responding to Peer Review}
The use of keywords is a useful way to aid the readers understanding. A glossary in the appendix to explain the key words would go even further to help allow non- mathematically minded people understand the paper. When using acronyms, it is useful to filly define them once before using them for the rest of the paper (What is OLS?).

Mentioning the omission of repeated measurements of a blackbird is useful in making the results reproducable. It is worth discussing how you did this and how it may effect the results. Cleaning the data according to species is a good decision, but a reference to how ring numbers are linked to species would be useful for the reader.

The model and conclusions are explained well and the IMRAD structure is well followed. To aid reader understanding in variable correlation, plots could be used. The model is well shown, but the use of the error term should be explained.

\begin{itemize}
	\item \textcolor[rgb]{1,0,0}{Thank you for your suggestions on the shortcomings of my article. For the OLS abbreviation, I have revised the introduction section. For the explanation of the error item, I put it in the discussion section. Limited by the length of the article, I put the visualization results and data cleaning code into the appendix.}
\end{itemize}

\newpage

Thank you for submitting your report; I really enjoyed reading it! Also, I like how you analysed the data. Below, I wrote some comments that I hope will help you improve your report.

I think that in the abstract there is slightly too much emphasis on the method of analysis. You may want to consider shortening the method section (of the abstract) and expanding a little on the background and what you found, and what the results mean. You can put the details concerning the dataset you used in the method section. To make space for this, I would recommend removing the last subsection of Table 1 (the omnibus, skew, and kurtosis part).

In the introduction, you could mention briefly some literature concerning blackbirds’ size in relation to their sex and age. It may also be useful, if you have space, to mention what you expect the results to be given the literature (or state hypotheses). Here are links to some references that may be useful: \url{https://onlinelibrary.wiley.com/doi/abs/10.1111/j.0030-1299.2006.14183.x https://doi.org/10.1080/00063656309476036 http://www.jstor.org/stable/2407788}

You mentioned that the independent variables do not explain variation in the dependent variable well, but, in the context of the blackbirds dataset (where the measurements were taken in a natural setting with no control over extraneous variables), I would consider the variation in the independent variables to explain the variation in the dependent variable relatively well. Specifically, in this case, I think that if the predictor variables ac- count for 45\% of the variance in the outcome variable, the predictors explain a substantial and meaningful proportion of the variance in the outcome. Overall, well-done.

\begin{itemize}
	\item \textcolor[rgb]{1,0,0}{Thank you for your advice and encouragement. For the results in the table, I will delete it. As well as, thank you for your supplementary references. I checked it and it really helped a lot. About the final question, I cannot point out this in the report due to the space limitation. I suppose that if the missing important explanatory variable has nothing to do with other explanatory variables, that is to say, the explanatory variable is orthogonal to the other explanatory variables, then the variance of unbiased estimator of parameter estimator is also unbiased, but it will cause biased estimation of constant term.}
\end{itemize}


\end{document}
